%% Generated by Sphinx.
\def\sphinxdocclass{report}
\documentclass[letterpaper,10pt,english]{sphinxmanual}
\ifdefined\pdfpxdimen
   \let\sphinxpxdimen\pdfpxdimen\else\newdimen\sphinxpxdimen
\fi \sphinxpxdimen=49336sp\relax

\usepackage[margin=1in,marginparwidth=0.5in]{geometry}
\usepackage[utf8]{inputenc}
\ifdefined\DeclareUnicodeCharacter
  \DeclareUnicodeCharacter{00A0}{\nobreakspace}
\fi
\usepackage{cmap}
\usepackage[T1]{fontenc}
\usepackage{amsmath,amssymb,amstext}
\usepackage{babel}
\usepackage{times}
\usepackage[Bjarne]{fncychap}
\usepackage{longtable}
\usepackage{sphinx}

\usepackage{multirow}
\usepackage{eqparbox}

% Include hyperref last.
\usepackage{hyperref}
% Fix anchor placement for figures with captions.
\usepackage{hypcap}% it must be loaded after hyperref.
% Set up styles of URL: it should be placed after hyperref.
\urlstyle{same}
\addto\captionsenglish{\renewcommand{\contentsname}{Contents:}}

\addto\captionsenglish{\renewcommand{\figurename}{Fig.\@ }}
\addto\captionsenglish{\renewcommand{\tablename}{Table }}
\addto\captionsenglish{\renewcommand{\literalblockname}{Listing }}

\addto\extrasenglish{\def\pageautorefname{page}}

\setcounter{tocdepth}{1}



\title{CritOpS Documentation}
\date{Apr 19, 2017}
\release{2.1.0}
\author{Andrew Johnson}
\newcommand{\sphinxlogo}{}
\renewcommand{\releasename}{Release}
\makeindex

\begin{document}

\maketitle
\sphinxtableofcontents
\phantomsection\label{\detokenize{index::doc}}



\chapter{Intro}
\label{\detokenize{intro::doc}}\label{\detokenize{intro:intro}}\label{\detokenize{intro:welcome-to-critops-s-documentation}}
This is the documentation for CritOpS, a Critical Optimization Search tool for use with NEWT{[}1{]}.
CritOpS is designed to iteratively modify inputs for NEWT to obtain a desired eigenvalue.
More documentation will be added before the final release of this code, including examples and validation testing.


\section{Setup}
\label{\detokenize{intro:setup}}\label{\detokenize{intro:id1}}
\begin{sphinxVerbatim}[commandchars=\\\{\}]
\PYG{n}{git} \PYG{n}{clone} \PYG{n}{https}\PYG{p}{:}\PYG{o}{/}\PYG{o}{/}\PYG{n}{github}\PYG{o}{.}\PYG{n}{com}\PYG{o}{/}\PYG{n}{drewejohnson}\PYG{o}{/}\PYG{n}{CritOpS}\PYG{o}{.}\PYG{n}{git}
\PYG{n}{cd} \PYG{n}{CritOpS}
\PYG{n}{python} \PYG{n}{setup}\PYG{o}{.}\PYG{n}{py} \PYG{n}{install}
\end{sphinxVerbatim}

The code currently requires \sphinxtitleref{python3} due to some formatting calls, and \sphinxtitleref{pandas} for some better data output.


\section{Usage}
\label{\detokenize{intro:id2}}\label{\detokenize{intro:usage}}
CritOpS can be run from the terminal while in the directory outside the critops folder with the command

\begin{sphinxVerbatim}[commandchars=\\\{\}]
\PYGZdl{} python critops \PYGZlt{}mainfile\PYGZgt{} \PYGZlt{}paramfile\PYGZgt{}
\end{sphinxVerbatim}

Where \textless{}mainfile\textgreater{} is a valid NEWT input file with some variables in place of valid values and \textless{}paramfile\textgreater{} is the
file that contains limits on iteration parameters, desired k-eff, and indicates the variable to be iterated upon.
See \sphinxcode{testing/iter\_tester.inp} and \sphinxcode{testing/param\_tester.txt} for one example case.

\begin{sphinxVerbatim}[commandchars=\\\{\}]
\PYG{n}{python3} \PYG{n}{CritOpS}\PYG{o}{.}\PYG{n}{py} \PYG{n}{inp\PYGZus{}file} \PYG{n}{param\PYGZus{}file} \PYG{p}{[}\PYG{o}{\PYGZhy{}}\PYG{n}{v}\PYG{p}{]} \PYG{p}{[}\PYG{o}{\PYGZhy{}}\PYG{n}{o} \PYG{n}{OUTPUT}\PYG{p}{]}

\PYG{n}{positional} \PYG{n}{arguments}\PYG{p}{:}
  \PYG{n}{inp\PYGZus{}file}              \PYG{n}{template} \PYG{n}{SCALE} \PYG{n+nb}{input} \PYG{n}{file}
  \PYG{n}{param\PYGZus{}file}            \PYG{n}{file} \PYG{n}{containing} \PYG{n}{parameters} \PYG{k}{for} \PYG{n}{operation}

\PYG{n}{optional} \PYG{n}{arguments}\PYG{p}{:}
  \PYG{o}{\PYGZhy{}}\PYG{n}{h}\PYG{p}{,} \PYG{o}{\PYGZhy{}}\PYG{o}{\PYGZhy{}}\PYG{n}{help}            \PYG{n}{show} \PYG{n}{this} \PYG{n}{help} \PYG{n}{message} \PYG{o+ow}{and} \PYG{n}{exit}
  \PYG{o}{\PYGZhy{}}\PYG{n}{v}\PYG{p}{,} \PYG{o}{\PYGZhy{}}\PYG{o}{\PYGZhy{}}\PYG{n}{verbose}         \PYG{n}{reveal} \PYG{n}{more} \PYG{n}{of} \PYG{n}{the} \PYG{n}{mystery} \PYG{n}{behind} \PYG{n}{the} \PYG{n}{operation}
  \PYG{o}{\PYGZhy{}}\PYG{n}{o} \PYG{n}{OUTPUT}\PYG{p}{,} \PYG{o}{\PYGZhy{}}\PYG{o}{\PYGZhy{}}\PYG{n}{output} \PYG{n}{OUTPUT}
                        \PYG{n}{write} \PYG{n}{status} \PYG{n}{to} \PYG{n}{output} \PYG{n}{file}
\end{sphinxVerbatim}

The parameter file controls iteration procedure and \sphinxtitleref{SCALE} execution.
Parameters that can be updated with the parameter file include
\begin{enumerate}
\item {} 
\sphinxtitleref{k\_target}: Desired value of k-eff to be obtained from the \sphinxtitleref{SCALE} runs

\item {} 
\sphinxtitleref{eps\_k}: Acceptable accuracy between \sphinxtitleref{k\_target} and each value of k-eff

\item {} 
\sphinxtitleref{iter\_lim}: Maximum number of times to run \sphinxtitleref{SCALE}

\item {} 
\sphinxtitleref{exe\_str}: Absolute path to your \sphinxtitleref{SCALE} executable.

\item {} 
\sphinxtitleref{var\_char}: Whatever character you want to use as a designator for the variables

\end{enumerate}

Currently, \sphinxtitleref{CritOpS} only supports one iteration variable, which is declared in the parameter file with:

\begin{sphinxVerbatim}[commandchars=\\\{\}]
\PYG{n}{iter\PYGZus{}var} \PYG{o}{\PYGZlt{}}\PYG{n}{var}\PYG{o}{\PYGZgt{}} \PYG{o}{\PYGZlt{}}\PYG{n}{start}\PYG{o}{\PYGZgt{}} \PYG{o}{\PYGZlt{}}\PYG{n+nb}{min}\PYG{o}{\PYGZgt{}} \PYG{o}{\PYGZlt{}}\PYG{n+nb}{max}\PYG{o}{\PYGZgt{}}
\end{sphinxVerbatim}

The input file should be a valid \sphinxtitleref{NEWT} input file, with some minor modifications.
There should exist certain values defined as variables preceeded by the \sphinxtitleref{var\_char},

\begin{sphinxVerbatim}[commandchars=\\\{\}]
cuboid 20 5   0  0 \PYGZhy{}\PYGZdl{}del\PYGZus{}z
\end{sphinxVerbatim}

Given some input and parameter files, the code will create and execute successive input files,
parse the outputs for the update k-eff, and then update the iteration variables.


\section{License}
\label{\detokenize{intro:license}}\label{\detokenize{intro:id3}}
MIT License

Copyright (c) 2017 Andrew Johnson

Permission is hereby granted, free of charge, to any person obtaining a copy
of this software and associated documentation files (the ``Software''), to deal
in the Software without restriction, including without limitation the rights
to use, copy, modify, merge, publish, distribute, sublicense, and/or sell
copies of the Software, and to permit persons to whom the Software is
furnished to do so, subject to the following conditions:

The above copyright notice and this permission notice shall be included in all
copies or substantial portions of the Software.

THE SOFTWARE IS PROVIDED ``AS IS'', WITHOUT WARRANTY OF ANY KIND, EXPRESS OR
IMPLIED, INCLUDING BUT NOT LIMITED TO THE WARRANTIES OF MERCHANTABILITY,
FITNESS FOR A PARTICULAR PURPOSE AND NONINFRINGEMENT. IN NO EVENT SHALL THE
AUTHORS OR COPYRIGHT HOLDERS BE LIABLE FOR ANY CLAIM, DAMAGES OR OTHER
LIABILITY, WHETHER IN AN ACTION OF CONTRACT, TORT OR OTHERWISE, ARISING FROM,
OUT OF OR IN CONNECTION WITH THE SOFTWARE OR THE USE OR OTHER DEALINGS IN THE
SOFTWARE.


\section{References}
\label{\detokenize{intro:references}}
{[}1{]}: M. D. DeHart, and S. Bowman, ``Reactor Physics Methods and Analysis Capabilities in SCALE,'' Nuclear Technology, Technical Paper vol. 174, no.2, pp. 196-213, 2011.


\chapter{Importing CritOps Iterator}
\label{\detokenize{external_iteration::doc}}\label{\detokenize{external_iteration:importing-critops-iterator}}

\section{External Usage}
\label{\detokenize{external_iteration:external-usage}}\label{\detokenize{external_iteration:external}}
The \sphinxcode{CritOpS} module can easily be imported into external processing scripts.
Presented here is an example of executing the iteration routine as a standalone process.

\begin{sphinxVerbatim}[commandchars=\\\{\}]
\PYG{k+kn}{from} \PYG{n+nn}{critops}\PYG{n+nn}{.}\PYG{n+nn}{iterator} \PYG{k}{import} \PYG{n}{itermain}
\PYG{k+kn}{from} \PYG{n+nn}{critops}\PYG{n+nn}{.}\PYG{n+nn}{utils} \PYG{k}{import} \PYG{n}{oprint}
\PYG{k+kn}{from} \PYG{n+nn}{critops}\PYG{n+nn}{.}\PYG{n+nn}{outputs} \PYG{k}{import} \PYG{n}{output\PYGZus{}landing}


\PYG{n}{critArgs} \PYG{o}{=} \PYG{p}{\PYGZob{}}
    \PYG{l+s+s1}{\PYGZsq{}}\PYG{l+s+s1}{k\PYGZhy{}target}\PYG{l+s+s1}{\PYGZsq{}}\PYG{p}{:} \PYG{l+m+mf}{9.98515E\PYGZhy{}01}\PYG{p}{,}
     \PYG{l+s+s1}{\PYGZsq{}}\PYG{l+s+s1}{eps\PYGZus{}k}\PYG{l+s+s1}{\PYGZsq{}}\PYG{p}{:} \PYG{l+m+mi}{1}\PYG{n}{E}\PYG{o}{\PYGZhy{}}\PYG{l+m+mi}{8}\PYG{p}{,}
    \PYG{l+s+s1}{\PYGZsq{}}\PYG{l+s+s1}{verbose}\PYG{l+s+s1}{\PYGZsq{}}\PYG{p}{:} \PYG{k+kc}{False}\PYG{p}{,}
    \PYG{l+s+s1}{\PYGZsq{}}\PYG{l+s+s1}{output}\PYG{l+s+s1}{\PYGZsq{}}\PYG{p}{:} \PYG{k+kc}{None}\PYG{p}{,}
    \PYG{l+s+s1}{\PYGZsq{}}\PYG{l+s+s1}{iter\PYGZus{}lim}\PYG{l+s+s1}{\PYGZsq{}}\PYG{p}{:} \PYG{l+m+mi}{15}\PYG{p}{,}
    \PYG{l+s+s1}{\PYGZsq{}}\PYG{l+s+s1}{exe\PYGZus{}str}\PYG{l+s+s1}{\PYGZsq{}}\PYG{p}{:} \PYG{l+s+s1}{\PYGZsq{}}\PYG{l+s+s1}{C:}\PYG{l+s+se}{\PYGZbs{}\PYGZbs{}}\PYG{l+s+s1}{Scale\PYGZhy{}6.2.1}\PYG{l+s+se}{\PYGZbs{}\PYGZbs{}}\PYG{l+s+s1}{bin}\PYG{l+s+se}{\PYGZbs{}\PYGZbs{}}\PYG{l+s+s1}{scalerte.exe}\PYG{l+s+s1}{\PYGZsq{}}\PYG{p}{,}
    \PYG{l+s+s1}{\PYGZsq{}}\PYG{l+s+s1}{k\PYGZhy{}id}\PYG{l+s+s1}{\PYGZsq{}}\PYG{p}{:} \PYG{l+s+s1}{\PYGZsq{}}\PYG{l+s+s1}{Input buckling}\PYG{l+s+s1}{\PYGZsq{}}\PYG{p}{,}
    \PYG{l+s+s1}{\PYGZsq{}}\PYG{l+s+s1}{k\PYGZhy{}col}\PYG{l+s+s1}{\PYGZsq{}}\PYG{p}{:} \PYG{l+m+mi}{9}\PYG{p}{,}
    \PYG{l+s+s1}{\PYGZsq{}}\PYG{l+s+s1}{stalequit}\PYG{l+s+s1}{\PYGZsq{}}\PYG{p}{:} \PYG{k+kc}{False}\PYG{p}{,}
\PYG{p}{\PYGZcb{}}

\PYG{n}{iter\PYGZus{}var} \PYG{o}{=} \PYG{p}{\PYGZob{}}\PYG{l+s+s1}{\PYGZsq{}}\PYG{l+s+s1}{buck}\PYG{l+s+s1}{\PYGZsq{}}\PYG{p}{:} \PYG{p}{(}\PYG{l+m+mf}{1.2E\PYGZhy{}03}\PYG{p}{,} \PYG{l+m+mf}{1.00E\PYGZhy{}03}\PYG{p}{,} \PYG{l+m+mf}{5.00E\PYGZhy{}3}\PYG{p}{)}\PYG{p}{\PYGZcb{}}

\PYG{n}{bFile} \PYG{o}{=} \PYG{l+s+s1}{\PYGZsq{}}\PYG{l+s+s1}{intHX5\PYGZus{}buck.inp}\PYG{l+s+s1}{\PYGZsq{}}

\PYG{n}{oprint}\PYG{p}{(}\PYG{l+s+s1}{\PYGZsq{}}\PYG{l+s+se}{\PYGZbs{}n}\PYG{l+s+s1}{Starting buckling iteration}\PYG{l+s+se}{\PYGZbs{}n}\PYG{l+s+s1}{\PYGZsq{}}\PYG{p}{,} \PYG{o}{*}\PYG{o}{*}\PYG{n}{critArgs}\PYG{p}{)}

\PYG{n}{iter\PYGZus{}vec}\PYG{p}{,} \PYG{n}{k\PYGZus{}vec}\PYG{p}{,} \PYG{n}{conv\PYGZus{}type} \PYG{o}{=} \PYG{n}{itermain}\PYG{p}{(}\PYG{n}{bFile}\PYG{p}{,} \PYG{n}{iter\PYGZus{}var}\PYG{p}{,}

\PYG{n}{output\PYGZus{}landing}\PYG{p}{(}\PYG{n}{iter\PYGZus{}vec}\PYG{p}{,} \PYG{n}{k\PYGZus{}vec}\PYG{p}{,} \PYG{n}{conv\PYGZus{}type}\PYG{p}{,} \PYG{o}{*}\PYG{o}{*}\PYG{n}{critArgs}\PYG{p}{)}
\end{sphinxVerbatim}


\section{Default Arguments}
\label{\detokenize{external_iteration:default-arguments}}\label{\detokenize{external_iteration:defaults}}
\noindent\begin{tabulary}{\linewidth}{|L|L|L|}
\hline
\sphinxstylethead{\relax 
\sphinxstylestrong{Parameter}
\unskip}\relax &\sphinxstylethead{\relax 
\sphinxstylestrong{Default}
\unskip}\relax &\sphinxstylethead{\relax 
\sphinxstylestrong{Note}
\unskip}\relax \\
\hline
\sphinxcode{eps\_k}
&
1E-4
&
Tightness on \sphinxtitleref{k} convergence
\\
\hline
\sphinxcode{k\_target}
&
1.0
&
Desired \sphinxtitleref{k-eff}
\\
\hline
\sphinxcode{iter\_lim}
&
50
&
Maximum number of iterations
\\
\hline
\sphinxcode{tiny}
&
1E-16
&
Numerical zero
\\
\hline
\sphinxcode{var\_char}
&
\sphinxcode{'\$'}
&
Character to identify iteration variables
\\
\hline
\sphinxcode{k-id}
&
\sphinxcode{'k-eff = '}
&
String to identify line containing \sphinxtitleref{k-eff}
\\
\hline
\sphinxcode{k-col}
&
2
&
Location of \sphinxtitleref{k-eff} in \sphinxcode{line.split()}
\\
\hline
\sphinxcode{stalequit}
&
\sphinxcode{True}
&
Terminate if \sphinxtitleref{k-eff} hasn't changed
\\
\hline
\sphinxcode{exe\_str}
&
C:\textbackslash{}SCALE-6.2.1\textbackslash{}bin\textbackslash{}scalerte.exe
&
Absolute path to \sphinxcode{SCALE} executable
\\
\hline\end{tabulary}



\chapter{iterator}
\label{\detokenize{iterator::doc}}\label{\detokenize{iterator:iterator}}\label{\detokenize{iterator:module-critops.iterator}}\index{critops.iterator (module)}
CritOpS

Andrew Johnson

Objective: Main file for controlling the iteration scheme

Functions:
\begin{quote}

iter\_main: Landing function that drives the iteration

makefile: Write the new output file using the value from iteration 
\_iter

update\_itervar: Simple function to update the iteration variables.

parse\_scale\_out\_eig: Read through the SCALE output file specified by 
\_ofile and return status and eigenvalue (if present)
\end{quote}
\index{itermain() (in module critops.iterator)}

\begin{fulllineitems}
\phantomsection\label{\detokenize{iterator:critops.iterator.itermain}}\pysiglinewithargsret{\sphinxcode{critops.iterator.}\sphinxbfcode{itermain}}{\emph{file\_name: str}, \emph{iter\_vars: dict}, \emph{kwargs: dict}}{}
Main function for controlling the iteration
\begin{quote}\begin{description}
\item[{Parameters}] \leavevmode\begin{itemize}
\item {} 
\sphinxstyleliteralstrong{file\_name} -- Name of template file

\item {} 
\sphinxstyleliteralstrong{iter\_vars} -- Dictionary of iteration variables and their starting, minima, and maximum values

\item {} 
\sphinxstyleliteralstrong{kwargs} -- Additional keyword arguments

\end{itemize}

\item[{Returns}] \leavevmode
k\_vec: List of progression of eigenvalue through iteration procedure

\item[{Returns}] \leavevmode
iter\_vecs: Dictionary of iteration and their values through iteration procedure

\item[{Returns}] \leavevmode

conv\_type - reason for exiting iter\_main

0: Accurately converged to target eigenvalue in specified iterations

1: iter\_var exceeded specified maximum twice

-1: iter\_var exceeded specified minimum twice

2: Reached iteration limit without reaching target eigenvalue

-2: Previous two k are close to similar


\end{description}\end{quote}

\end{fulllineitems}

\index{update\_itervar() (in module critops.iterator)}

\begin{fulllineitems}
\phantomsection\label{\detokenize{iterator:critops.iterator.update_itervar}}\pysiglinewithargsret{\sphinxcode{critops.iterator.}\sphinxbfcode{update\_itervar}}{\emph{iter\_vars: dict}, \emph{iter\_vec: dict}, \emph{kvec: (\textless{}class `list'\textgreater{}}, \emph{\textless{}class `tuple'\textgreater{})}, \emph{ktarg: float}, \emph{**kwargs}}{}
Simple function to update the iteration variables.
Currently set up for a positive feedback on the variables.
I.e. increasing each iteration variable will increase k
\begin{quote}\begin{description}
\item[{Parameters}] \leavevmode\begin{itemize}
\item {} 
\sphinxstyleliteralstrong{iter\_vars} -- Dictionary of iteration variables and their minima/maxima

\item {} 
\sphinxstyleliteralstrong{iter\_vec} -- Dictionary of iteration variables and their values through the iteratio procedure

\item {} 
\sphinxstyleliteralstrong{kvec} -- Vector of eigenvalues

\item {} 
\sphinxstyleliteralstrong{ktarg} -- Target eigenvalue

\end{itemize}

\item[{Returns}] \leavevmode
status
status = 0 if the updated value is inside the intended range
status = 1 if the desired updated value is greater than the specified maximum of the parameter
and the max value is used as the updated value
status = -1 if the desired updated value is less than the specified maximum of the parameter
and the minimum value is used as the updated value

\end{description}\end{quote}

\end{fulllineitems}

\index{parse\_scale\_out\_eig() (in module critops.iterator)}

\begin{fulllineitems}
\phantomsection\label{\detokenize{iterator:critops.iterator.parse_scale_out_eig}}\pysiglinewithargsret{\sphinxcode{critops.iterator.}\sphinxbfcode{parse\_scale\_out\_eig}}{\emph{\_ofile: str}, \emph{**kwargs}}{}
Read through the SCALE output file specified by \_ofile and return status and eigenvalue (if present)
\begin{quote}\begin{description}
\item[{Parameters}] \leavevmode
\sphinxstyleliteralstrong{\_ofile} -- SCALE .out file

\item[{Returns}] \leavevmode

Status, eigenvalue

status = True if output file exists and eigenvalue was extracted
status = False if output file exists but no eigenvalue was found (possible error in input file syntax)
exit operation if no output file found


\end{description}\end{quote}

\end{fulllineitems}



\chapter{readinputs}
\label{\detokenize{readinputs::doc}}\label{\detokenize{readinputs:readinputs}}\label{\detokenize{readinputs:module-critops.readinputs}}\index{critops.readinputs (module)}
CritOps

Andrew Johnson

Objective: Read the inputs, update global variables, and check for proper variable usage

Functions:
\begin{quote}

check\_inputs: make sure values in global\_parameters are good for running
read\_param: Read the parameter file and update values in globalparams
readmain: Main driver for reading and processing the input files
\end{quote}
\index{readmain() (in module critops.readinputs)}

\begin{fulllineitems}
\phantomsection\label{\detokenize{readinputs:critops.readinputs.readmain}}\pysiglinewithargsret{\sphinxcode{critops.readinputs.}\sphinxbfcode{readmain}}{\emph{tmp\_file}, \emph{param\_file}, \emph{kwargs: dict}}{}
Main driver for reading and processing input files.
\begin{quote}\begin{description}
\item[{Parameters}] \leavevmode\begin{itemize}
\item {} 
\sphinxstyleliteralstrong{tmp\_file} -- Template input file

\item {} 
\sphinxstyleliteralstrong{param\_file} -- Parameter file

\item {} 
\sphinxstyleliteralstrong{kwargs} -- Additional arguments
- verbose (True) - status updates
- output (None) - print to screen
- Plus additional iteration parameters

\end{itemize}

\item[{Returns}] \leavevmode
List of valid template file lines and dictionary of interation variables
Updates kwargs based on values in param\_file

\end{description}\end{quote}

\end{fulllineitems}

\index{read\_param() (in module critops.readinputs)}

\begin{fulllineitems}
\phantomsection\label{\detokenize{readinputs:critops.readinputs.read_param}}\pysiglinewithargsret{\sphinxcode{critops.readinputs.}\sphinxbfcode{read\_param}}{\emph{\_pfile}, \emph{**kwargs}}{}
Read the parameter file and update values in kwargs
\begin{quote}\begin{description}
\item[{Parameters}] \leavevmode
\sphinxstyleliteralstrong{\_pfile} -- Parameter file

\item[{Returns}] \leavevmode
iter\_vars: Dictionary of iteration variables and their starting, minima, and maximum values

\item[{Returns}] \leavevmode
updated keyword arguments

\end{description}\end{quote}

\end{fulllineitems}

\index{check\_inputs() (in module critops.readinputs)}

\begin{fulllineitems}
\phantomsection\label{\detokenize{readinputs:critops.readinputs.check_inputs}}\pysiglinewithargsret{\sphinxcode{critops.readinputs.}\sphinxbfcode{check\_inputs}}{\emph{temp\_lines: list}, \emph{iter\_vars: dict}, \emph{**kwargs}}{}
Run over the inputs and make sure things are good for operation

\end{fulllineitems}



\chapter{outputs}
\label{\detokenize{outputs::doc}}\label{\detokenize{outputs:module-critops.outputs}}\label{\detokenize{outputs:outputs}}\index{critops.outputs (module)}
NRE6401 - Molten Salt Reactor

CritOpS

Objective: Functions for reading SCALE output files and writing output files

Functions:
\begin{quote}

parse\_scale\_output: Parse through the SCALE output file and return status
\end{quote}
\index{output\_landing() (in module critops.outputs)}

\begin{fulllineitems}
\phantomsection\label{\detokenize{outputs:critops.outputs.output_landing}}\pysiglinewithargsret{\sphinxcode{critops.outputs.}\sphinxbfcode{output\_landing}}{\emph{iter\_vecs: dict}, \emph{k\_vec: (\textless{}class `list'\textgreater{}}, \emph{\textless{}class `tuple'\textgreater{})}, \emph{\_outtype: int}, \emph{**kwargs}}{}
Write the output file according to the type of output
\begin{quote}\begin{description}
\item[{Parameters}] \leavevmode\begin{itemize}
\item {} 
\sphinxstyleliteralstrong{iter\_vecs} -- Dictionary with iteration variables and their values through the procedure

\item {} 
\sphinxstyleliteralstrong{k\_vec} -- Vector of eigenvalues

\item {} 
\sphinxstyleliteralstrong{\_outtype} -- Flag indicating the reason the program terminated
- 0 Nothing went wrong
- 1 Desired update value for iteration parameter twice exceeded the maximum value from the parameter file
- -1 Desired update value for iteration parameter twice exceeded the minimum value from the parameter file
- 2 Exceeded the total number of iterations allotted
- -2 No excessive change in eigenvalue

\end{itemize}

\item[{Returns}] \leavevmode


\end{description}\end{quote}

\end{fulllineitems}



\chapter{Indices and tables}
\label{\detokenize{index:indices-and-tables}}\begin{itemize}
\item {} 
\DUrole{xref,std,std-ref}{genindex}

\item {} 
\DUrole{xref,std,std-ref}{modindex}

\item {} 
\DUrole{xref,std,std-ref}{search}

\end{itemize}


\renewcommand{\indexname}{Python Module Index}
\begin{sphinxtheindex}
\def\bigletter#1{{\Large\sffamily#1}\nopagebreak\vspace{1mm}}
\bigletter{c}
\item {\sphinxstyleindexentry{critops.iterator}}\sphinxstyleindexpageref{iterator:\detokenize{module-critops.iterator}}
\item {\sphinxstyleindexentry{critops.outputs}}\sphinxstyleindexpageref{outputs:\detokenize{module-critops.outputs}}
\item {\sphinxstyleindexentry{critops.readinputs}}\sphinxstyleindexpageref{readinputs:\detokenize{module-critops.readinputs}}
\end{sphinxtheindex}

\renewcommand{\indexname}{Index}
\printindex
\end{document}